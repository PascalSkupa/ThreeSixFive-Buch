\section{Icons}

Es sollen Icons erstellt werden, die die vierzehn Allergene, die siebzehn No-Gos und die fünf Diäten in dem Formular auf der Webseite abbilden. Alle Icons in einem Icon-Set sollen optisch harmonieren. Der Gestaltungsstil, die Liniendicke und die Form der Zeichnungen soll ähnlich sein und zum gesamten Bild der Webapplikation passen. \\

Als Grundlage wurde ein grundlegendes Designkonzept genommen. Später wurden weitere moderne Designtrends dazu gemischt und mit eigenen, zum Projektthema passenden Merkmalen versehen. 


\subsection{Grundlegendes Designkonzept} 
In den letzten 20 Jahren kristallisierten sich zwei Konzepte heraus: Skeuomorphismus und Flat.\cite{SkeuomorphismusvsFlat}

\paragraph{Skeuomorphismus} 
Skeuomorphismus ist ein Bereich des Webdesigns, in dem die digitalen Elemente ähnlich zu ihren physischen Vorbildern abgebildet werden. Ein Synonym für Skeuomorphismus ist Realismus. Ein typisches Beispiel ist das Design von Apple iOS, bis zur Version iOS 7. Dort wurde die Stoff- und Ledertextur öfters eingesetzt, unter anderem sah das Icon für das Leseprogramm wie ein Buch aus Papier mit einer Lederbindung aus. \\

Zu Beginn der Entwicklung des Designs und der digitalen Technologien standen Unternehmen wie Apple, IBM und Microsoft vor einer Hürde. Sie mussten das Gerät und die zuvor unbekannte Benutzeroberfläche dem User zum ersten Mal vorstellen. Ihre Aufgabe war, mithilfe von Grafiken, richtigen Beschriftungen und bekannten Farben, den Usern klarzumachen, welche Funktion, welcher Button hat und wann man ihn drücken muss. Die realistische Darstellung der Elemente hat der damaligen Generation bei der Lernkurve geholfen.\\

Nach Ablauf der Zeit wurden die neuen Techniken gewöhnlich und es 
begann die Entwicklung der nativen Apps. Im Jahr 2009, als Softwareunternehmen anfingen, massiv Handy-Apps herzustellen, wurde Skeuomorphismus als veraltet und unübersichtlich angesehen. Die Reaktions- und Ladezeiten der Anwendungen mussten steigen und es musste immer mehr auf die Benutzerfreundlichkeit geachtet werden.\\ 

In diesem Stadium erschien der „Flat“ Stil. Microsoft betrat diesen Weg zuerst und leistete einen großen Beitrag zur Entwicklung des „flachen“ Stils, dessen Popularität in weiteren Jahren immens zunahm.


\paragraph{Flat}
Das „flache“ Design\cite{dasflacheDesign} strebt keine Übertragung des Volumens an, daher steht die zweidimensionale Visualisierung im Mittelpunkt. Dies bedeutet, dass keine Schatten, Reflexe oder komplizierte Texturen bei der Entwicklung verwendet werden. Die Einzige Ausnahme ist das Einsetzen von langen Schatten, die mehr Tiefe in das Bild bringen und das Objekt vom Hintergrund abheben. Flat ist ein Paradebeispiel für den Minimalismus. Dabei entfernen sich Webdesigner von komplexen Visualisierungsansätzen und vereinfachen das Motiv bis auf das Nötigste. Es werden einfache Formen verwendet und klare Konturen gemacht, die die Leichtigkeit des flachen Designs betonen sollen. \\

Die Farbpalette besteht in der Regel aus mehreren kontrastreichen Farben. Nicht die Anzahl der vorkommenden Farben, sondern richtige und harmonierende Farbkombinationen spielen eine wichtige Rolle. Die beliebtesten Farben sind Primär- und Sekundärfarben, also die Grundfarben (Cyan, Magenta, Yellow, Black) und ihre Mischformen. Auch Retro-Farben, wie Lachs und Lila werden häufig bei der Farbwahl miteinbezogen.\\

Mittlerweile gibt es eine Erweiterung für das flache Design. Designer verwenden zusätzlich zu den einfachen und prägnanten Elementen im zweidimensionalem Raum, ein bis zwei Techniken für Tiefe und Perspektive. Ein bekanntes Beispiel ist das Material Design, auch „Semi-flat“ genannt, von Google. Dort kommen auch Schatten und sanfte Animationen vor. Näheres zu Material Design befindet sich im Kapitel "8.1 Material Design". 

\paragraph{Entscheidung} 
Für die Gestaltung der Icons für das Projekt \textit{ThreeSixFive} fiel die Entscheidung auf das „flache“, bzw. „Semi-flache“ Designkonzept. Die Grafiken sollen nach aktuellen und modernen Richtlinien gestaltet werden und zum Material Design der gesamten Anwendung passen. Die vielen unterschiedlichen Produktarten sollen klar erkennbar dargestellt werden, dafür werden Schatten, kontrastreiche Farben sowie Farbverläufe gebraucht. 

\subsection{Aktuelle Icondesigntrends} 

Bekannte Webdesignfirmen veröffentlichen jährlich eine Liste von aktuellen und modernen Richtlinien. Das ist eine Liste von zusammengefassten Trends aus dem Jahr 2019. \cite{Icondesigntrends}



\paragraph{Helle Farben}
Helle und herausstechende Farben werden im Webdesign immer häufiger angewandt. Dabei ist es wichtig, eine angemessene und harmonische Kombination zu finden. Speziell dafür haben manche Programme Techniken und Tools.\\
Das Illustrationsprogramm, Adobe Illustrator bietet das Bedienfeld „Farbhilfe“ \cite{farbhilfeillustrator}, das Vorschläge für mögliche Farbharmonien angibt. Zu der ausgewählten Farbe werden komplementäre, teilkomplementäre, linke komplementäre, rechte komplementäre, analoge, triadische und viele weitere Farbkombinationen angezeigt. Zwei Farben sind dann komplementär, wenn sie sich im Farbkreis gegenüber liegen, zum Beispiel Blau und Gelb, Rot und Cyan, Grün und Magenta. Mischt man sie zusammen, ergeben sie Schwarz oder Weiß, abhängig von dem Farbmodell. \\Analoge Farben sind drei Farben, die nebeneinander liegen. Diese Kombination wirkt harmonisch und besitzt je nach der Wahl der Farben eine Temperatur, kalt oder warm. \\Für das triadische Farbschema werden drei Farben verwendet, die im gleichen Abstand zueinander im Farbkreis liegen, dabei wird eine als Hauptfarbe und die anderen als Akzentfarben verwendet.\\
Zusätzlich werden für die Vorgeschlagenen Farben unterschiedliche Schattierungsmöglichkeiten präsentiert. Aus diesen Vorschlägen lassen sich Farbgruppen generieren, die je nach Wunsch in einem weiteren Feld verändert werden können. Adobe bietet schon vordefinierte Farbgruppen, wie Erdfarben, Wasserfarben und Metallfarben an.

\paragraph{Kreise und Rundungen}
In der Natur kommen keine geraden Linien vor. Daher wirken geschwungene und weiche Formen für einen User viel natürlicher. Sie sorgen für ein angenehmes Gefühl und vermitteln den Eindruck von Fließfähigkeit und Beweglichkeit. 

\paragraph{Simplizität}
Einfache Formen, Linien und Farben werden kombiniert, um vereinfachte Darstellungen für Icons zu erstellen. Die Verwendung von vielen Farben und Linien bei kleinen Grafiken ist anstrengend für das menschliche Auge. Durch das Einschränken der Designkomplexität werden die Bilder besser erkennbar und verständlicher.

\paragraph{Gradient}
Weiche Übergänge von gesättigten oder gedämpften Farben werden immer öfter eingesetzt. Damit können originelle visuelle Effekte, Volumen und neue Farbtöne erzeugt werden. Dieser Trend kehrte erst in den letzten Jahren zurück, da früher beim „Flat“ Design nur einfache Farben verwendet wurden und Gradiente für altmodisch gegolten haben. 

\paragraph{Flat mit mehr Volumen}
Das traditionelle „Flat“ Design wird durch das Hinzufügen von Schatten voluminöser, während die grundlegenden Minimalismuskonzepte, die das flache Design populär gemacht haben, beibehalten werden.



\subsubsection{Skizzen und Konzeptauswahl} 

Es wurden viele unterschiedliche Designrichtlinien in Betracht gezogen und viele Skizzen angefertigt. Die besten Ideen wurden kombiniert und mit mehreren Motiven und Hintergründen ausprobiert.

\begin{figure}[H] \centering \includegraphics{skizzen.png} \caption{Iconskizzen} \end{figure} 


\paragraph{Entscheidung} 

Die Icons für die No-Gos, Allergene und Diäten wurden viele bunte Farben verwendet. Zur Auswahl des Hintergrunds kam zuerst immer die komplementäre Farbe zu der Hauptfarbe im Vordergrund infrage, falls diese schon verwendet wurde, so wurde eine andere gewählt. Somit gab es keine vordefinierte Farbpalette und kein Hintergrund kommt doppelt vor. \\
Um das Motiv hervorzugeben, wurde ein langer Schatten eingebaut. Dieser ist immer zwei Schattierungen dunkler als die Hintergrundfarbe und ist im selben Winkel, was einen einheitlichen, aber gleichzeitig andersfarbigen Schatten bei allen Icons ermöglicht. Für den Hintergrund wurde ein Kreis, ohne Ränder, gewählt. \\
Allgemein wurden wenige Linien verwendet und viel Wert auf die einfache Gestaltung gelegt. Gradiente wurden bei schwereren Bildern benutzt, um dem Produkt mehr Volumen zu geben und somit eine bessere Assoziierung zu erreichen. 

\subsection{Grafik- und Exportformat}
Bevor die Arbeitsumgebung eingerichtet und Grafiken erstellt werden können, müssen das Grafik- und Exportformat festlegt werden.


\paragraph{Grafikformat}
Es gibt zwei Hauptformate für das Erstellen von Webgrafiken, Raster und Vektor. Die Besonderheit des Rasterformats besteht darin, dass es wie eine Mosaik aus kleinen Teilen - Pixeln - besteht. Je höher die Auflösung, desto größer ist die Anzahl der Pixel pro Flächeneinheit. Pixelbilder werden verwendet, um einen sanften Übergang von Farben und ihren Schattierungen, zu vermitteln. Die häufigste Anwendung ist die Fotobearbeitung, das Erstellen von Collagen und Flyer.\\


Die Logik eines Vektorbildes ist völlig anders. In Vektorgrafikobjekten gibt es sogenannte Bezugspunkte, zwischen denen sich Kurven befinden. Die Krümmung dieser Kurven wird durch mathematische Formeln beschrieben. Vektorgrafiken werden häufig beim Drucken verwendet, für Broschüren, Flugblätter, Visitenkarten und alle möglichen Produkte mit Text, Logo, Mustern, Ornamenten, also für alles, was nicht die exakte Übertragung von Farbverläufen erfordert und mit Kurven beschrieben werden kann. Es ist so gut wie unmöglich mit Formeln und Punkten so gute Farbübergänge, wie mit einer Pixelgrafik, zu vermitteln. Der größte Vorteil von Vektorbildern ist, dass sich die Bildqualität auch bei starker Vergrößerung nicht ändert, da jedes Bild einzeln berechnet wird.\\

Für die \textit{ThreeSixFive} Anwendung müssen 36 Icons erstellt werden, die in einer Liste auf der Webseite platziert werden. Die Größe ist nicht genau bekannt, da das Bild für den Laptop und für das Handy unterschiedlich skaliert werden muss. Der Stil ist einfach und flach gestaltet, es werden nur wenige Farbverläufe verwendet. 

\paragraph{Exportformat}
Eine Vektorgrafik muss in einem Format exportiert werden, das dafür geeignet ist.  Dafür wurde das SVG-Format (skalierbare Vektorgrafik) entwickelt. Es ist geeignet zum Entwickeln und Beschreiben von zweidimensionalen Vektorbildern sowie für das Hinzufügen eines Skripts und animierter 3D-Bilder. Da sich das SVG-Format auf Vektorbilder bezieht, ist es möglich, jeden Teil davon zu vergrößern, ohne die Bildqualität zu beeinträchtigen. Sein Vorteil ist, dass der Text als Text von der Suchmaschine erkannt und daher indiziert wird.

\subsection{Umsetzung}

\subsubsection{Einrichten der Arbeitsumgebung}

Für die Erstellung von Vektorgrafiken und Illustrationen hat Adobe den Adobe Illustrator entwickelt. Die Arbeitsfläche muss vor der Entwicklung eingestellt werden.\\
Es wurden 37 kleine Artboards erstellt. In dem ersten werden die Icons erstellt und später in die anderen verschoben. Zusätzlich wurden Bezugslinien und ein Lineal zum Positionieren der Motive und der Schatten erstellt.\\
\begin{figure}[H] \centering \includegraphics{artboards.png} \caption{Einrichtung des Illustratorfensters und des Lineals zur Messung der Schattenwinkel} \end{figure}

\subsubsection{Allergene}

\begin{figure}[H] \centering \includegraphics{icons_allergens.png} \caption{Die fertigen Allergen-Icons} \end{figure}

Es wurden Icons für Fisch, Laktose, Soja, Lupine, Nüsse, Gluten, Senf, Sesam, Erdnüsse, Sellerie, Sulfite, Weichtiere, Eier und Krebstiere erstellt. 

\subsubsection{No-Gos}

\begin{figure}[H] \centering \includegraphics{icons_nogo.png} \caption{Die fertigen No-Go-Icons} \end{figure}

Es wurden Icons für Schweinefleisch, Kohl, Lakritze, Meeresfrüchte, Soja, Tomate, Lammfleisch, Broccoli, Sojasoße, Rosinen, Tofu, Fisch, Sojadrink, Rindfleisch, Nüsse, Pilze und Sojajoghurt erstellt.\\
Es wurden dieselben Icons für Nüsse, Fisch und Soja genommen wie bei den Allergenen um die Darstellung von ein und derselben Sache nicht unterschiedlich zu machen. Das Milch-Icon wurde etwas abgeändert, bekam eine neue Aufschrift und das Soja-Symbol. Das Soja-Symbol wurde immer wieder bei sojahaltigen Produkten eingebaut. Das Fish- und Krebstiere-Symbol wurde bei den Meeresfrüchten verwendet.

\subsubsection{Diäten}

\begin{figure}[H] \centering \includegraphics{icons_diets.png} \caption{Die fertigen Diät-Icons} \end{figure}

Es wurden Icons für Diäten, wie Milchfrei, Glutenfrei, Proteinreich, Kalorienarm und Kohlenhydratarm.\\
Diese bekamen zusätzlich noch eine Aufschrift, damit der User genau erkennt, nach welchem Ernährungskonzept dieses Diät aufgebaut ist. Ein Apfel ist zwar Kalorienarm, dennoch steht er nicht direkt für eine kalorienarme Ernährung, ebenso wie Fleisch für Proteinreich oder Brot für Kohlenhydratarm. 



