\section{Material Design}

Google erstellte 2014 unter dem Codenamen „Quantum Paper“ eine Art Regelwerk für das Design von Google Applikationen. Das Ziel war es das Design so einheitlich zu halten das für den Nutzer sofort erkennbar sein sollte, dass es sich um eine von Google veröffentlichte Software handelt. Gleichzeitig sollten durch das gleichbleibende Design die Navigation und Bedienbarkeit, einer Webseite oder Applikation, intuitiv zwischen allen Plattformen funktionieren. Der Name Material Design\cite{MaterialDesign} stammt von dem Grundgedanken, dass verschiedene Materialien, wie Stoffe und andere Texturen, natürliche Eigenschaften haben, welche einem Nutzer direkt auffallen. Auf diesem analogen Grundgedanken basierend wurde Material Design an Tinte und Papier angelehnt, da diese dem Menschen ein natürliches Gefühl übermitteln.

Die Grundidee hinter dem Design entspringt aus dem Wunsch eine moderne und intuitiv bedienbare Benutzeroberfläche zu bieten. Diese hat sowohl simpel als auch ansprechend und kreativ auszusehen.

Die größten Faktoren für Zufriedenheit der Nutzer sind die Gestaltung, Aufteilung und Präsentation einer Applikation. Deshalb hält man sich an die Material Design Richtlinien. Laut Material Design ist die Designsprache eine Vereinigung aus klassischen Designprinzipien und den Möglichkeiten, welche durch technische Innovationen geschaffen werden.

Die Einheitlichkeit des Designs ist ein wichtiger Aspekt. Das Nutzererlebnis soll unabhängig von Plattform, Gerät und Eingabemethode gleich sein.
Unter allen diesen Vorgaben und Richtlinien ein einzigartiges und ansprechendes Design zu entwerfen ist die Aufgabe, die es umzusetzen gilt.


\section{Corporate Identity}

Als Corporate Identity\cite{CorporateIdentity} bezeichnet man all jenes was Nutzer oder Außenstehende von einem Unternehmen sehen. Alles, was sie wahrnehmen und mit dem Unternehmen in Verbindung gebracht werden kann. In dem Fall von \textit{ThreeSixFive} handelt sich bei der Applikation selbst um das Unternehmen und somit sind jegliche Designs der Applikation der Corporate Identity gleichzusetzen. Die Corporate Identity soll das Unternehmen von anderen abheben und so für ein einzigartiges Auftreten sorgen. Sie soll wie die Persönlichkeit einer realen Person unverwechselbar sein und gleichzeitig einen hohen Wiedererkennungswert haben. Hierbei handelt es sich nicht bloß um visuelles Auftreten, sondern auch um Umgangsformen, firmeninterne Regeln und andere Merkmale, durch die sich das Unternehmen auszeichnet.

Bei \textit{ThreeSixFive} wurde die Corporate Identity nicht zwanghaft entwickelt. Diese entstand erst im Laufe der Planung und der Umsetzung des Projektes. Immer mehr Aspekte wurden durch vermehrtes Auftreten automatisch in die Projekt Struktur integriert und bildeten so unsere Corporate Identity. Die Corporate Identity besteht aus den Unterkategorien Corporate Behavior, Corporate Communication und für \textit{ThreeSixFive} am relevantesten das Corporate Design.


\subsection{Corporate Design}

Das Corporate Design\cite{CorporateIdentity} ist ein Teil der Corporate Identity. Hierbei handelt es sich jedoch speziell um das visuelle Auftreten des Unternehmens. Es werden markante visuelle Elemente definiert und regelmäßig angewandt, um einen hohen Wiedererkennungswert beim Konsumenten zu gewährleisten. Ebenso werden Regeln für die Designsprache eines Unternehmens festgelegt, welche sich durch das gesamte äußere Erscheinungsbild ziehen sollen. Elemente wie Logos, Grafiken, Farben, Slogans, Werbungen, Schriftarten und Texturen kommen zum Einsatz, um dies zu ermöglichen. Das Corporate Design sollte über alle Applikationen wie aber auch Printmedien, Präsentationen und Werbungen gleich sein. Durch die Wiederholung genannter Elemente prägt sich das Design beim Nutzer ein und sorgt für eine klare Identität des Unternehmens.

Bei \textit{ThreeSixFive} wurde das Corporate Design vorwiegend durch die Gestaltung der Webseite definiert. Für diese wurden die Farben, Schriftarten und Grafiken ausgewählt. Durch viele technische und persönliche Vorgaben wurde ein fester Rahmen geboten, in welchem das Konzipieren stattfand. Technische Voraussetzungen, wie die Responsivität der Webseite spielten dabei eine genauso große Rolle wie auch persönliche Wünsche, wie Farbvorlieben oder ein helles Design. Eine Möglichkeit ein Corporate Design festzulegen ist zudem das Erstellen von Material Themes. Diese Themes umfassen nahezu alle Aspekte des Corporate Designs. \textit{ThreeSixFive} verzichtete auf das Erstellen von Material Themes aufgrund des Verwendens der Bibliothek PrimeNG, welche schon ein grobes Design Konzept für einzelne Komponenten bietet.

\subsubsection{Farben}

Die Farben\cite{Farben} einer Webseite verbessern den Wiedererkennungswert einer Webseite und somit auch die Corporate Identity. Ansprechende Farben beeinflussen wie Nutzer das Gesehene interpretieren, ebenso stark, wie dies der Inhalt selbst tut. Die Verknüpfung von Farben und Marken/Applikationen kann vielzählig nachgewiesen werden, bei Coca-Cola zum Beispiel ist die Farbe Rot mindestens genauso ausschlaggebend wie der Geschmack des Produktes selbst. Die Farbe hilft aus der Masse herauszustechen und ruft zugleich gewünschte Gefühle hervor. Gewisse Farben kommen mit gewissen Emotionen, Rot in diesem Fall wird mit Liebe, Aufregung und Passion in Verbindung gebracht. Diese durch Farben ausgelösten Emotionen werden verwendet, um eine persönlichere Verbindung zwischen dem potenziellen Kunden und einem Produkt zu erzeugen.

Gerade bei Webseiten, welche im Zusammenhang mit Essen stehen, sind Farben wichtig. Die Farben beeinflussen die Stimmung des Nutzers und bieten einen größeren Anreiz die Seite zu öffnen und auch länger dort zu verweilen.

Die richtigen Farben auszuwählen war ein wichtiger Schritt im definieren des Designs für \textit{ThreeSixFive}. Die Zielgruppe und die Funktionen der Applikation spielten bei der Wahl eine signifikante Rolle. Um die richtige Wahl der Farben zu treffen, muss zuerst ein gewisses Grundverständnis für Farben vorhanden sein. Hierzu ist es wichtig, Farbmodelle zu kennen und zu verstehen. Farbmodelle dienen dazu Farben zu beschreiben. Es ist unmöglich, für Millionen von Farben, Namen zu haben. Zwei der relevantesten Modelle sind das RGB-Modell und das CMYK-Modell.

Das RGB-Modell wird meist bei digitalen Designs verwendet. Es beschreibt, für die drei Grundfarben Rot, Grün und Blau, einen Wert zwischen 0 und 255. Diese Werte mischen sich dann im gegebenen Verhältnis zu der Wunschfarbe zusammen.

Das CMYK-Modell wird überwiegend im analogen Segment, wie zum Beispiel dem Druck verwendet. Es beschreibt Farben basierend auf ihren jeweilig prozentualen Anteilen von Cyan (C), Magenta (M), Gelb (Y) und Schwarz (K).

Für \textit{ThreeSixFive} sollten die Farben einen gewissen genießbaren Eindruck machen, deshalb wurden ausschließlich Farben verwendet, welche auch in natürlichen Lebensmitteln vorkommen.

Der Prozess lief folgendermaßen ab:\\

Es wurden Farben gefunden, welche laut Material Design zu einem harmonischen Ganzen beitragen und genügend Kontrast bieten. Dann wurden diese durch Reverse-Lookup über Google.com in Bildern welche Tags wie „Essen“, „Tasty“ oder „Delicious“ hatten wiedergefunden. Die Bilder zeigen nun Lebensmittel, die ähnlichen Farben wie die Wunschfarben aufweisen. Nun wurden die Wunschfarben nur noch an eben genau diese natürlichen Farben angepasst.

\begin{figure}[H] \centering \includegraphics{farben.png} \caption{Die fertigen Farben} \end{figure}

\subsubsection{Logo}

Das Logo\cite{Logo} ist der einfachste und direkteste Weg sich als Marke, Produkt oder Applikation visuell auszudrücken. Es übermittelt die Grundidee der Applikation auf eine simple und dennoch aussagekräftige Art und Weise. Die für die Applikation konzipierte Designsprache ist in dem Logo erkennbar und reflektiert so die Identität der Applikation.

\paragraph{Die Erstellung mittels Adobe Photoshop und Adobe Illustrator}

Das Logo sollte die Applikation repräsentieren und leicht mit Kochen und Rezepten in Verbindung gebracht werden können.

Mit der Umsetzung wurde begonnen, indem auf Papier grobe Gedanken in Form von Skizzen aufgezeichnet wurden und so wurde mit diversen Ideen experimentiert, um eine genauere Vorstellung, von den Ideen, zu bekommen. In diesem Prozess entstand die Idee einen Pfannenwender und Kochlöffel als typische Symbole des Kochens zu verwenden.
Nun wurden der Pfannenwender und der Kochlöffel in Photoshop mithilfe von groben Formen erstellt. Die beiden Symbole wurden überkreuzt um ein „Wappen“-ähnliches Aussehen zu erreichen. Der Entwurf wurde weiter verfeinert und angepasst. In diesen Prozess wurde das gesamte Team involviert, sodass jedes Mitglied seine Gedanken und Ideen einbringen konnte. Nachdem es zu der Zufriedenheit aller verbessert wurde, hat man den Entwurf zu Adobe Illustrator exportiert.
Adobe Illustrator bietet hervorragende Tools zur Erstellung von Vektorgrafiken. Letzte kleine Verbesserungen wie die exakte Positionierung wurden vorgenommen. Der Entwurf wurde in Illustrator nun vektorisiert, indem er mit Pfaden nachgezogen wurde. Nach letzten Überprüfungen wurde das fertige Logo exportiert. Hierzu wurden diverse Formate abgespeichert.

\input{text/bildtypentabelle}

\begin{figure}[H] \centering \includegraphics{logo.png} \caption{Das fertige Logo} \end{figure}

\subsubsection{Schrift}

Richtig angewandte Typografie\cite{Schrift} ist ein mächtiges Mittel zur Übermittlung von jeglichen Informationen. Die Typografie sollte den Inhalt unterstützen und das Aufnehmen von Informationen erleichtern. Sie spiegelt die Bedürfnisse der Applikation wider, diese sind nicht nur von stilistischer Natur, sondern auch in Abhängigkeit der angewandten Technik und Funktionalitäten der Web-Applikation. All diese Aspekte fließen in die Entscheidung der Font Wahl ein und bestimmen somit indirekt einen wichtigen Teil des Designs.

Das Schriftbild und somit der verwendete Web-Font ist ein Teil des Designs und somit Teil der Corporate Identity. Das Ziel ist es das Design und den Inhalt so eindeutig wie und auch so effizient wie möglich darzstellen.

Um eine klare Hierarchie der Schrift zu erzeugen, werden Skalierungen\cite{Skalierung} der Schriftgröße, Stärke und des Zeichenabstandes vorgenommen. Laut Material Design umfasst diese Skalierungen 13 verschiedene Stile. Die Stile sind als CSS-Klassen verwendbar und sollten auch ordnungsgemäß angewandt werden, um eine klare Strukturierung zu gewährleisten.

Eine weitere Möglichkeit Überschriften vom Inhalt abzuheben ist es mehrere Fonts miteinander zu kombinieren. So entsteht eine eindeutige visuelle Abgrenzung, welche es dem Nutzer erleichtert durch die Seite zu navigieren und Inhalte aufzunehmen. Hierauf wurde bei \textit{ThreeSixFive} verzichtet, denn die Navigation ist aufgrund der nicht vorhandenen Unterüberschriften eindeutig erkennbar.  Es wird ausschließlich ein Font mit verschiedenen Skalierungen verwendet.

\begin{figure}[H] \centering \includegraphics{schrift.png} \caption{Schriftskalierungen nach Material Design (Lizenz zum Bild: \cite{SchriftskalierungenBild})} \end{figure}

\paragraph{Google-Fonts}
In Angular ist standardmäßig eine Sammlung von Web Fonts mit dem Namen Google-Fonts\cite{GoogleFonts}. Diese enthält eine Vielzahl an Fonts, in über 135 verschiedenen Sprachen, welche für die Anwendung im Web und Druck optimiert sind. Diese werden in Zusammenarbeit von Nutzern und einem Kerndesigner Team erstellt und gewartet. Google-Fonts verbessern die Ladezeiten von Web-Applikationen, indem es Fonts standardmäßig zwischenspeichert. Das Zwischenspeichern funktioniert Webseiten übergreifend, das heißt man muss eine Google Font, von egal welcher Webseite, nur einmal laden und kann diese dann für alle anderen Webseiten, welche den gleichen Font nutzen aus dem Zwischenspeicher laden. Der Google-Fonts Server sendet zudem die kleinstmögliche Datei basierend auf den, von dem Nutzer verwendeten, Browser und den jeweiligen unterstützten Technologien. Das zielt darauf ab die gleiche Qualität und Integrität eines Designs / einer Webseite weltweit zur Verfügung zu stellen, unabhängig von Standort und Internetbandbreite.

In \textit{ThreeSixFive} wird die Font „Roboto“\cite{Roboto} verwendet, diese ist eine der von Material-Design empfohlenen Fonts. „Roboto“ hat geometrische Formen, welche gerade Linien und offene Kurven darstellen, dennoch basiert der Zeichensatz auf einem mechanischen Grundgerüst. Laut den Google-Fonts Statistiken wird Roboto auf über 24 Millionen Webseiten verwendet und gehört so zu den populärsten Google Fonts.

\begin{figure}[H] \centering \includegraphics{font.png} \caption{Der Font Roboto (Lizenz zum Bild: \cite{RobotoBild})} \end{figure}

\section{Angular Frontend}
Angular\cite{AngularDokumentation} ist ein Framework, um interaktive Komponenten für eine Webseite zu erstellen. Es wurde als umfangreiches JavaScript Framework von der Google LLC entwickelt. Angular ist Open-Source-Software und ist damit eines der größten Frontend-Webapplikationsframeworks. Es ist in TypeScript geschrieben und aufgrund der eindeutigen Unterteilung der einzelnen Komponenten eignet es sich perfekt für \textit{ThreeSixFive}.

Angular basiert auf der Model-View-Controller Methode. Entwickler haben dieses Modell seit langem verwendet jedoch ist Angular das erste JavaScript Framework, welches darauf aufbaut.

Dieses Prinzip teilt die Entwicklung auf drei Ebenen auf.

\paragraph{„View/Template“} als das Sichtbare, welches die Nutzererfahrung ausmacht und mit HTML und CSS/SCSS geschrieben wird.

\paragraph{„ViewModel/Component“} jede Komponente definiert eine Klasse, welche die Applikationsdaten und -Logik für ein Template enthalten.

\paragraph{„Service/Injector“} für Daten und Logik, welche keinem speziellem Template zugeordnet sind und welche man Komponenten-übergreifend verwenden möchte. Diese müssen mittels Injektor den Komponenten als Dependency übergeben werden.
\\

\begin{figure}[H] \centering \includegraphics{angularoverview.png} \caption{Überblick der Schnittstellen in Angular (Lizenz zum Bild: \cite{SchnittstellenBild})} \end{figure}

Das Model-View-Controller Modell erlaubt die Wiederverwendung von Templates und Komponenten und ist somit ideal für die Applikation \textit{ThreeSixFive}. Außerdem erlaubt das „two-way data-binding“ zwischen Template und Component ständige Updates zwischen beiden Ebenen. Ohne Angular müsste der Entwickler sich selbst um den ständigen Austausch zwischen Nutzer Eingaben und Aktion/Werte-Veränderungen in der Logik kümmern. Solch eine push und pull Logik selbst zu schreiben ist umständlich und fehleranfällig.

Im Folgenden Diagramm wird aufgezeigt, wie Angular mit dem DOM kommuniziert.

\begin{figure}[H] \centering \includegraphics{databinding.png} \caption{Überblick des "Data-Bindings" (Lizenz zum Bild: \cite{DatabindingBild})} \end{figure}
Es werden vier verschiedene Formen des data-bindings gezeigt.\\
 Jede Form hat eine Richtung:\\
\begin{itemize}
\item Zu dem DOM
\item Vom DOM
\item Beidseitig sowohl vom DOM als auf zum DOM
\end{itemize}

Angular modifiziert direkt die DOM-Struktur einer Seite und fügt sich nicht in das HTML ein. Dies führt zu einer besseren Performance der Applikation.
In dieser Umgebung, welche von Angular geboten wird, gilt es nun die Applikation umzusetzen.

\section{Design in Angular}
Angular\cite{AngularDokumentation} 7 Applikationen werden mit Standard CSS gestaltet. Dies erlaubt die Anwendung von bekannten Praktiken wie Selektoren und Media-Queries in Angular Applikationen.

Jede Angular Komponente kann ein HTML-Template und ein CSS erhalten. Das CSS definiert die Selektoren, Regeln und Media-Queries welche in der Komponente benötigt werden. Das Stylesheet wird in der Komponente verlinkt.

\section{CSS}

„Cascading Style Sheets“\cite{CSS} ist eine Formatvorlage für Auszeichnungssprachen. Diese wird für die Gestaltung von HTML-Dokumenten oder XML-Dokumenten verwendet. Sie beschreibt, wie einzelne Elemente gerendert werden sollen. CSS dient als eine der wichtigsten deklarativen Programmiersprachen der Webentwicklung.

\subsection{Selektoren}
Elemente, welche nun gestaltet werden, sollen mittels Selektoren angesprochen werden. Die mit diesem Selektor verknüpften Regeln werden so dem jeweiligen Element zugewiesen. Zudem können Selektoren miteinander kombiniert werden. Hier gibt es einige verschiedene Kombinationen, wie die zum Beispiel die „Vererbende Kombination“, welche zwei oder mehrere Selektoren, die hintereinander aufgerufen werden, hierarchisch miteinander kombiniert. Dadurch bekommt jedes untergeordnete Element auch die Werte des Übergeordneten zugewiesen. Ein weiteres Beispiel ist die allgemeine „Geschwister Kombination“, diese erlaubt es Regeln nur dann zu übergeben, wenn das gewünschte Element einem bestimmten anderen Element als „Geschwister-Element“ folgt. Voraussetzung ist hierbei das beide Elemente Teil des gleichen „Eltern-Elementes“ sind. Zu der Verwendung trennt man nun beide Selektoren durch eine Tilde bei ihrem Aufruf.

Sollten Elemente multipel angesprochen werden, überschreibt die neuere Regel immer die zuvorkommende. Dies ist wichtig um Regeln für Elemente gezielt bei bestimmten Aktionen zu überschreiben wie das Hovern, Aktivieren oder Fokussieren eines Elementes.

\subsection{Layout}
CSS-Layouts\cite{CSSLayout} sind die durch CSS gebotenen Möglichkeiten, Elemente auf einer Seite zu platzieren, wie diese sich zueinander und zu dem Sichtfenster verhalten. Hierzu werden verschieden Tools zur Verfügung gestellt um die optimale Darstellung unter gegebenen Voraussetzungen, wie die zum Beispiel die Displaygröße, zu gewährleisten. Große Schlagwörter des modernen CSS-Layouts sind Flexbox und CSS-Grid.

\subsubsection{CSS-Box-Modell}
Beim Rendern einer Webseite interpretiert der Webbrowser jegliche Elemente als rechteckige Boxen\cite{BoxModel}. Mittels CSS werden anschließend Eigenschaften wie Farbe oder Hintergrund festgelegt. Zugleich wird jede Box nach bestimmten CSS Regeln positioniert. Die Elemente haben eine bestimmte Höhe und Breite, welche gemeinsam die Grenzen um das Element bilden. Diese können mit Befehlen wie \codeword{width}, \codeword{min-width}, \codeword{max-width}, \codeword{height}, \codeword{min-height}, und \codeword{max-height} verändert werden.

Zwischen dem eigentlichen Element und diesen äußeren Grenzen liegt das „Padding“. Diesem kann man ebenso statische oder variable Werte zuweisen, um einen Abstand zwischen Element und Rahmen zu erzeugen.

Das „Padding“ wird von einem Rahmen umgeben. Dem Rahmen können fixierte Werte zugewiesen werden, ebenso kann dieser aber auch weggelassen werden.

Der am weitesten außen gelegene Bereich nennt sich „Margin“. „Margin“, beschreibt den Abstand, welcher, ausgehend von dem Rahmen eines Elementes, zu anderen Elementen gehalten wird. Der „Margin“ wird durch die Befehle \codeword{margin-top}, \codeword{margin-right}, \codeword{margin-bottom} und \codeword{margin-left} definiert. Die festgelegten Werte werden zwischen verschiedenen Boxen geteilt.


\subsubsection{CSS-Layoutmodus}
Der Layoutmodus, auch kurz Layout genannt, ist der Umgang von Boxen/Elementen untereinander, zu ihren Nachbar-Elementen und ihren Eltern-Elementen. Über die Zeit wurden diverse CSS-Layoutmodi entwickelt und zu CSS hinzugefügt.\\
Die wichtigsten Layoutmodi umfassen: \\
\begin{itemize}
\item Das Blocklayout, welches darauf abzielt Dokumente darzustellen.
\item Das Tabellenlayout, welches entwickelt wurde, um Tabellen nachbilden zu können.
\item Das Flexboxlayout, welches das Umbrechen und Umpositionieren von Elementen dynamisch ermöglicht und somit ideal ist, um Seiten in unterschiedlichen Größen darzustellen.
\item Das CSS-Gridlayout, welches ein fixiertes Raster vorgibt, nachdem sich die Elemente positionieren.
\end{itemize}

\subsection{Einheiten}
CSS bietet diverse Einheiten\cite{Einheiten}, welche es erlauben Längen nach unterschiedlichen Parametern darstellen zu lassen. Viele CSS Befehle verlangen eine „Länge“ als Wert. Zudem gibt es Befehle, welche eine negative Länge erlauben, zum Beispiel „margin“ oder „padding“. Generell wird zwischen zwei verschiedenen Arten von Längeneinheiten unterschieden, die absoluten und die relativen Längeneinheiten.

Absolute Längeneinheiten werden genauso angezeigt, wie sie deklariert werden, hierbei werden keine anderen Faktoren berücksichtigt. Dies hat den Nachteil, dass Elemente eine fixierte Größe haben und das verwendete Endgerät nicht berücksichtigt wird. Die absoluten Längeneinheiten sind: Centimeter (cm), Millimeter (mm), Inch (in), Pixel (px) und Points (pt).

Relative Längeneinheiten sind abhängig von anderen Werten. Dies verbessert die Skalierbarkeit von Webseiten durch die Möglichkeit Elemente in unterschiedlichen Größen anzuzeigen. Die wichtigsten relativen Längeneinheiten sind: em (Abhängig von der verwendeten Schriftgröße inerhalb des Elementes), vw (abhängig von der Breite des Bildausschnittes), vh (abhängig von der Höhe des Bildausschnittes) und \% (abhängig vom Eltern-Element).

\subsection{CSS Probleme}
Ein Problem, welches mit CSS aufgetreten ist, ist das Überschreiben von definierten CSS-Regeln durch die Library PrimeNG bei dem Gestalten der PrimeNG-Komponenten. PrimeNG lädt die sein Stylesheet mithilfe von Node.js in Angular, diese Stylesheets haben standardmäßig die höchste Priorität. Die Reihenfolge von den geladenen Stylesheets lässt sich in Angular nicht ändern, weshalb die benötigten Regeln und Media-Queries in der „styles.scss“-Datei definiert werden mussten. Die „styles.scss“-Datei ist das Globale Stylesheet von Angular. Dieses wirkt global auf alle Komponenten und hat in Angular die höchste Priorität. Das heißt, anstatt die Gestaltung in den CSS-Dateien der jeweiligen Komponenten zu definieren wurde dies global definiert, um das Überschreiben durch das PrimeNG-Stylesheet zu umgehen.

Ein weiteres Problem war Ausnahmen in bestimmten Komponenten zu definieren. In einem bestimmten Element wurden Styles definiert, welche für eben dieses Element und alle Kind Elemente gelten sollte, bis auf ein einziges Element, welches anders gestaltet gehört oder nur ein paar der Regeln mit seinen Nachbar-Elementen teilt. Anfangs wurden diese Elemente mit einem ID-Selektor angesprochen, was jedoch, aufgrund der vielen ID-Selektoren, zu einer starken Unübersichtlichkeit geführt hat. Hier kommt die CSS-Pseudoklasse „:not(X)“\cite{notcss} zum Einsatz. Hierbei beschreibt der Selektor „X“ die Ausnahme, welche die Styles nicht erhalten soll. So ist es möglich, schnell und übersichtlich, Ausnahmen von Definitionen zu definieren, ohne eine hohe Redundanz innerhalb eines Stylesheets aufzuweisen.
\\

Syntax: \codeword{:not(selector) \{ Stileigenschaften \} }

\section{PrimeNG}
PrimeNG ist eine Sammlung von User-Interface Komponenten für Angular. Der Code zu jeder einzelnen PrimeNG Komponente ist Open-Source und steht unter der MIT-Lizenz. Über 80 Komponenten werden angeboten und viele davon finden Anwendung in \textit{ThreeSixFive}. Die Komponenten können durch diverse Themes in ihrem Aussehen angepasst werden, auf dies wird jedoch verzichtet und stattdessen ein eigenes Design verwendet. PrimeNG bietet durch seine Vielfalt an Komponenten eine hohe Variation und viel Anwendungsmöglichketen welche an verschiedenen Stellen in der Applikation zum Einsatz kommen. Die vorgefertigten Komponenten erleichtern das Erstellen eines User-Interface und beschleunigen den Entwicklungsprozess.

Es folgen einige Komponenten, die näher erläutert werden.

\subsection{Card}
Cards gehören zu den Panel-Elementen und dienen der Darstellung von Content und Eingabefeldern. Cards sind flexible Container und passen sich mit ihrer Größe den enthaltenen Elementen an. Sie heben sich durch eine Schattierung von dem Hintergrund ab und sorgen somit für eine visuelle Abgrenzung zwischen verschiedenen Bereichen.

\begin{figure}[H] \centering \includegraphics{cardbsp.png} \caption{Code eines Card Beispiels} \end{figure}
\begin{figure}[H] \centering \includegraphics{cardout.png} \caption{Ausgabe eines Card Beispiels} \end{figure}
\begin{figure}[H] \centering \includegraphics{card365.png} \caption{Verwendung von Cards bei \textit{ThreeSixFive}} \end{figure}

\subsection{SelectButton}
SelectButtons gehören zu den Input-Elementen und somit zu den Eingabefeldern. Sie dienen der einfachen oder mehrfachen Auswahl von diversen Werten mithilfe von Buttons. Jeder SelectButton ist mit einem Wert und einer Sammlung von Möglichkeiten verbunden. Diese Sammlung von Möglichkeiten wird mithilfe eines Arrays im Controller der Komponente verwirklicht. SelectButtons helfen dabei mehr Kontrolle über die Darstellung und Gruppierung von Eingabeknöpfen zu haben.

\begin{figure}[H] \centering \includegraphics{selectbsp.png} \caption{Code eines SelectButton Beispiels} \end{figure}
\begin{figure}[H] \centering \includegraphics{selectout.png} \caption{Ausgabe eines SelectButton Beispiels} \end{figure}
\begin{figure}[H] \centering \includegraphics{select365.png} \caption{Verwendung von SelectButtons bei \textit{ThreeSixFive}} \end{figure}

\subsection{Grid CSS}
Grid CSS ist in PrimeNG enthalten und stellt ein ressourcenschonendes responsives Grid-Layout zu Verfügung. Mithilfe des Grid-Layout lassen sich Applikationen leicht responsiv gestalten und somit für Mobile-Endgeräte und Desktops zugleich entwickeln. Ein Grid CSS basiert auf einem zwölf Spalten Layout. Jede Spalte wird mit der CSS-Klasse „ui-g-*“ definiert. Diese muss innerhalb einer „ui-g“ Klasse liegen, welche als Reihe dient. Die Spaltengröße wird durch die Endzahl angegeben zum Beispiel „ui-g-4“ würde bedeuten, dass 4/12 einer Reihe dieser Spalte zu Verfügung stehen.

Für die Responsivität werden zusätzliche CSS-Klassen für verschiedene Display Größen hinzugefügt. Jede unterstützte Größe hat einen anderen Media-Query für sich definiert. Die verschiedenen Größen werden mit den Präfixen \codeword{ui-sm-*}, \codeword{ui.md-*}, \codeword{ui-lg-*}, und \codeword{ui-xl-*} angesprochen.

\begin{figure}[H] \centering \includegraphics{mediaq.png} \caption{Media-Query Klassen} \end{figure}

\begin{figure}[H] \centering \includegraphics{gridbsp.png} \caption{Code eines Grid Beispiels} \end{figure}
\begin{figure}[H] \centering \includegraphics{gridout.png} \caption{Ausgabe eines Grid Beispiels} \end{figure}

\subsection{PrimeNG Icons}
Icons\cite{CSSIcons} werden verwendet, um den textbasierten Inhalt mit visuellen Abbildungen in Form von Piktogrammen zu komplementieren. Dies führt zu einem besseren Nutzungserlebnis und hilft bei der Navigation. Zudem steigert die sinnvolle Verwendung von Icons auch den Wiedererkennungswert einer Webseite.

Ein einfacher Weg CSS Icons in eine Web-Applikation zu implementieren sind Icon Bibliotheken wie „Material Icons“ oder „Font Awesome Icons“. Solche Icon Bibliotheken werden als Stylesheet in eine Web-Applikation eingebunden. Die Icons sind Vektoren, welche mit CSS bearbeitet werden können und somit können auch CSS-Regeln, wie „size“ oder “color“ verwendet werden, um jeweilige Icons anzupassen. \textit{ThreeSixFive} verwendet die in PrimeNG enthaltene Bibliothek PrimeIcons. Diese müssen nicht zusätzlich implementiert werden.

Um Icons nun zu verwenden, wird ein „inline“ HTML-Element mit einem CSS Klassen Selektor versehen.
