Nach dem Zweiten Weltkrieg gab es eine Nahrungsknappheit und es wurde immer schwerer sich Rezepte zu überlegen und die Einkäufe der kommenden Woche zu planen. Zur Hilfe kamen einem die letzten Seiten der alten Kochbücher, dort fand man einen Speiseplan für das ganze Jahr. Jeder Tag war genau durchgeplant und hat viel Abwechslung, trotz der schwierigen Lebenssituation, geboten.\\

Heutzutage sind Lebensmittel viel leichter verfügbar und trotzdem zerbrechen sich Familien oft darüber den Kopf was gekocht werden soll. Zusätzlich hat jedes Familienmitglied zusätzlich unterschiedliche Ernährungspräferenzen und Allergien. Von dieser Idee inspiriert wurde \textit{ThreeSixFive} entwickelt, um den modernen Alltag zu vereinfachen.\\



Die Webanwendung \textit{ThreeSixFive} generiert mithilfe des internen Algorithmus einen Speiseplan, der individuell an jede Familie angepasst werden kann. Die einzelnen Wochen, Tage und Speisen können personalisiert werden. Zum Beispiel kann in einem Formular eingestellt werden, wie viele Mahlzeiten man am Tag generiert haben möchte, ob jemand aus der Familie bestimmte Allergien oder Unverträglichkeiten hat und ob man einem bestimmten Ernährungskonzept, wie „high-protein“ oder „glutenfrei“, folgen möchte. Alle Präferenzen können in einem privaten Account festgelegt und gespeichert werden.\\
Eine zusätzliche Funktion ist die Einkaufsliste. Alle für die kommende Woche benötigten Zutaten werden in eine Liste übertragen und können beim Einkaufen abgehakt werden. Sowohl die Einkaufsliste als auch die Wochenpläne lassen sich exportieren und können auch offline angesehen werden. \\
Aufgrund der riesigen Auswahl an Rezepten in englischer Sprache, wird die Anwendung vollständig in Englisch geschrieben. 
